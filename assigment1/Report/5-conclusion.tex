À notre échelle, nous avons pu découvrir un problème d’organisation concret. Lors du découpage du travail en différentes tâches, il a fallu concilier les intérêts et les affinités de chacun afin de répartir les différentes étapes de manière optimale.\\

Les interviews nous ont permis de découvrir la différence de point de vue qu’il y a entre les employés et les employeurs. En nous immisçant dans l’univers d’une entreprise multinationale, nous avons pu affronter un univers professionnel dans lequel le côté financier et donc, le profit est parfois plus important que le social.

Cependant, la seconde interview avec le représentant syndical, nous a montré qu’une protection existait pour la cheville ouvrière de l’entreprise et que les employés avaient un pouvoir de réaction, de décision dans leur entreprise. C’est encourageant de savoir que, contrairement aux États-Unis où le libéralisme sauvage est la norme, un employeur belge doit prendre en compte l’aspect humain de ses salariés. Malgré les vagues de licenciements et le fait que ceux qui sont restés ont dû - parfois de manière drastique - augmenter leur charge de travail pour compenser les départs, c’est rassurant de voir que la représentation syndicale a poussé l’entreprise à proposer des solutions de réinsertion dans leurs plans de carrière, comme dit précédemment. Selon le directeur des ressources humaines d’UCB que nous avons également interviewé, ce choix d’offrir une structure d’accompagnement à la réinsertion professionnelle est une volonté émanant aussi de la direction. Suite à cette déclaration, nous nous sommes posé la question de l’importance de la gestion de l’image de l’entreprise confrontée à un plan de restructuration.\\

D’un point vue plus personnel, ce travail, différent de ceux que nous avons l’habitude de faire dans notre programme de bachelier en informatique, nous a permis de changer de contexte et de voir la situation dans des domaines professionnels qui ne seront fort probablement pas directement les nôtres. Nous espérons que ce travail reflète la réalité qui nous a été dépeinte par les deux acteurs sociaux que nous avons rencontré.
